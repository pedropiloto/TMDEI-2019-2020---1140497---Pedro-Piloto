% ************ Chapter 6 ************
%\renewcommand{\chaptername}{Chapter}

\chapter{Avaliação e Experimentação}
\label{cap:6}
\emph{
Apesar de ser expectável a existência de avaliações ao longo do processo de desenvolvimento, existem algumas grandezas que apenas são possíveis de ser avaliadas em fases mais avançadas. 
Depois do desenvolvimento de um projeto, é necessário avaliá-lo de forma a perceber se os problemas identificados foram resolvidos e se ao mesmo tempo a solução apresenta qualidade suficiente.
Neste contexto, o presente capítulo, consiste na avaliação da solução desenvolvida tanto a nível técnico como a nível funcional, definindo experiências e testes e analisando os resultados dos mesmos.}

\section{Experiências e Testes}
Neste contexto, uma experiência ou teste, consiste na medição ou avaliação de uma determinada grandeza, utilizando uma metodologia ou técnica específica de modo a validar ou refutar uma hipótese.

Assim, para a realização de um teste é preciso definir as grandezas, as hipóteses e as metodologias de avaliação.

\subsection{Grandezas a Avaliar}
Tendo em conta que o desenvolvimento deste projeto consiste na migração da plataforma Solid para uma arquitetura orientada a micro-serviços de forma a incrementar a sua escalabilidade, existem duas potenciais grandezas a avaliar:
\begin{itemize}
    \item Qualidade da implementação - A qualidade da implementação é crucial para tornar viável a aceitação da solução como uma contribuição para a comunidade em volta do Solid. A qualidade deverá ser mensurada tendo em conta métricas como resultados de software.
    \item Escalabilidade - É importante perceber se de facto a solução implementada é mais escalável que a plataforma actual.
\end{itemize}

\subsection{Hipóteses}
Uma hipótese consiste numa afirmação que se quer corroborar através de testes estatísticos utilizando grandezas identificadas.

No contexto deste projeto, foram definidas quatro hipóteses:
\begin{itemize}
    \item Sucesso 100\% dos testes de software e cobertura superior a 80\%;
    \item Capacidade de cada micro-serviço deve ser de pelo menos 75\% da suportada pela plataforma actual (monolítico) para uma determinada carga, mantendo o tempo máximo de resposta de cada pedido inferior a 400 milissegundos.
\end{itemize}

\subsection{Metodologias de Avaliação}
As metodologias de avaliação consistem na forma como serão verificadas as hipóteses definidas. A metodologia a utilizar irá depender da hipótese e da grandeza em causa.

A primeira hipótese, tendo em conta a qualidade da solução implementada, deve recorrer a testes de software, nomeadamente testes unitários e testes de integração que são executados através de ferramentas específicas para este efeito. Tendo em conta que esta metodologia é directa, não deverá ser necessário tratamento estatístico.

A segunda hipótese consiste em corroborar que cada micro-serviço do sistema, sob uma determinada carga, consegue suportar pelo menos 75\% do \emph{throughput} da plataforma actual (monólito), mantendo o tempo de resposta de cada pedido abaixo de 400 milissegundos. Para esta hipótese, deverão ser realizados testes de performance aos diferentes micro-serviços e à plataforma actual. Estes testes deverão ser executados recorrendo a ferramentas para este efeito (como por exemplo JMeter) e os resultados devem ser apontados de forma a poderem ser tiradas conclusões. É importante referir que no caso dos micro-serviços, é expectável que os pedidos realizados durante os testes de carga passem pela \emph{API-Gateway}, de forma a que o cenário se aproxime o mais possível da realidade.

\section{Resultados}