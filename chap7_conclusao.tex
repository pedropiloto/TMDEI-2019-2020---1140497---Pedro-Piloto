% ************ Chapter 7 ************
%\renewcommand{\chaptername}{Chapter}

\chapter{Conclusões e trabalho futuro}
\label{cap:7}

\section{Visão geral}

Esta dissertação permitiu conhecer um conceito novo e disruptivo para redesenhar a \emph{Web} que está a ser desenvolvido e alavancado por diferentes projetos e comunidades (\emph{c.f.} secção \ref{section_modelos_descentralizados}).

Numa primeira fase as diferentes alternativas com maior potencial foram estudadas com vista a perceber de que forma respondiam a desafios tecnológicos como escalabilidade, segurança e privacidade, este estudo foi refletido tanto no estado da arte do presente documento (\emph{c.f.} capítulo \ref{chapter_estado_arte}), como também em contribuições públicas (\emph{c.f.} secção \ref{section_contribuicoes_investigacao}).

O estudo das diferentes alternativas permitiu escolher um projeto para aprofundar numa perspetiva mais técnica com vista a criar um impacto positivo de escalabilidade ao mesmo tempo que suscitou uma mentalidade de contribuir para uma comunidade \emph{open-source}. Tratando-se de um projeto cujo objetivo passa por revolucionar a \emph{Web}, estas contribuições implicaram pensar fora da caixa num mundo já repleto de fortes alicerces implantados por grandes empresas e grandes engenheiros da área da informação e tecnologias de comunicações (ICT).

O \emph{Solid} foi o projeto escolhido e sobre o qual incidiu uma grande parte do trabalho relatado nesta dissertação. O esforço no sentido de migrar a arquitetura monolítica obrigou a perceber todo o trabalho até então realizado e distribuído por mais de 20 repositórios no \emph{github} (\emph{c.f.} secção \ref{section_contribuicoes_investigacao}). Para além do \emph{solid-server} proposto (\emph{c.f.} secção \ref{section_arquitetura_proposta}), houve também trabalho desenvolvido no sentido de criar provas de conceito com interface gráfica, uma delas com co-autoria desenvolvida no sentido de colaboração com uma tese de dissertação do \emph{\acrshort{MIT}} e outra como forma de suporte à apresentação do presente documento.

\section{Questões de investigação e contribuições}
Desde o início que o problema foi decomposto em questões mais simples com vista a colmatar as diferentes indefinições.

\newhalfpara

\textbf{Q1: {Como pode o \emph{Solid} ser escalado horizontal e verticalmente?}}

O escalamento vertical não deverá ser um problema, uma vez que corresponde a aumento de recursos que suportam a instância a executar.

Por outro lado, o escalamento horizontal é um problema na arquitetura atual, na medida em que cada nova execução do \emph{Solid} corresponde a um nova instância totalmente independente das outras que existem no mundo (\emph{c.f.} figura 4.1).

Uma possível solução consiste em embeber os diferentes serviços (no caso da arquitetura orientada a micro-serviços) em \emph{containers} (utilizando por exemplo a tecnologia \emph{docker}) e executar estes \emph{containers} através de uma ferramenta orquestração (como por exemplo \emph{kubernetes}).

É importante denotar que para isto ser possível é também necessário que os recursos estáticos estejam disponíveis e de alguma forma sincronizados entre as diferentes instâncias de um determinado serviço(\emph{c.f.} secção \ref{section_arquitetura_proposta}).

\newhalfpara

\textbf{Q2: É possível existir apenas uma implantação do \emph{Solid} escalada de forma descentralizada e anónima?}

Uma implantação única e global do \emph{Solid} seria uma possível evolução para o conceito. Esta evolução está dependente da migração para uma arquitetura orientada a micro-serviços, da adoção de um mecanismo de descoberta de novas instâncias para determinado serviço (\emph{c.f.} capítulo \ref{desenho})

\newhalfpara

\textbf{Q3: O mecanismo de autenticação utilizado atualmente pode ser mantido tendo em conta a migração para uma arquitetura orientada a micro-serviços?}

O mecanismo de autenticação utilizado atualmente pelo \emph{Solid} é o \emph{Open-ID Connect} em combinação com a atualização de \emph{tokens} de acesso em formato \emph{\acrshort{JWT}} (\emph{c.f.} secção \ref{section_estado_arte_solid_authentication}).

O fluxo de autenticação do protocolo \emph{Open-ID connect}, cria uma clara separação entre o componente responsável por gerir a autenticação (\emph{Identity Provider} e o componente que serve recursos com base na autenticidade do \emph{token} de acesso (\emph{Resource Server} (\emph{c.f.} figura 2.1). O \emph{token} garante que a validação de autenticidade não depende do componente emissor, e desta forma contribui para o baixo acoplamento característico numa arquitetura orientada a micro-serviços (\emph{c.f.} secção \ref{subsection:jwt}).


\section{Resultados}
o trabalho desenvolvido demonstra que é possível migrar a atual arquitetura monolítica do \emph{Solid} para uma arquitetura orientada a micro-serviços capaz de garantir escalabilidade horizontal e vertical dos diferentes serviços.

Esta nova arquitetura não compromete nem as funcionalidades de negócio nem a segurança, sendo possível manter todos os mesmos protocolos que suportam o projeto hoje em dia.

\section{Trabalho futuro e limitações}

No seguimento do trabalho desenvolvido é possível, também, denotar funcionalidades que devem ser implementadas no futuro, bem como limitação que devem ser colmatadas.

\subsection{Dependência sistema de ficheiros}
O sistema \emph{Storage Web} e o \emph{Storage Consumer} tem uma dependência do sistema de ficheiros como forma de persistência que pode ser um possível \emph{bottleneck} em termos de escalabilidade desta nova arquitetura, devendo ser estudada a migração para um sistema de ficheiros virtual ou mesmo uma diferente forma de persistência mais robusta.
\subsection{Orquestração de \emph{containers}} O projeto desenvolvido foi desenhado no sentido de todos os diferentes projeto poderem ser executados em \emph{containers}, os próximos passos passam pela integração com uma ferramenta de orquestração de \emph{containeirs} (exemplo: \emph{kubernetes}
\subsection{\emph{Continuous Integration} e \emph{Continuous Delivery}}
A adoção em grande escala deste projeto implica o desenvolvimento de \emph{pipelines} robustas de integração e entrega contínua.

\subsection{Considerações finais}
A disrupção foi certamente o combustível de motivação para o desenvolvimento desta dissertação, que, numa perspetiva global cumpriu os objetivos inicialmente propostos bem como conclusões e desafios futuros com potencial de contribuição positiva para uma sociedade cibernética mais protegida e livre.