% ************ Chapter 1 ************
%\renewcommand{\chaptername}{Chapter}
\chapter{Introdução}
\label{cap:1}
\emph{Este capítulo incide na apresentação contextualizada do tema que servirá de base a todo trabalho desenvolvido. Neste sentido, é importante explicar o contexto do tema e realçar as motivações que alicerçaram todo o estudo à sua volta. Não menos importante é explicar os objetivos propostos para esta tese, por forma a balizar e mensurar o trabalho desenvolvido.
Em suma, este capítulo deverá servir de resumo àquilo que o leitor deve esperar encontrar nos restantes capítulos do presente documento.}


\section{Contextualização}
Numa era em que o principal ativo das empresas é a informação, a ponto de existirem empresas que trocam serviços altamente valiosos por informação pessoal (ex: Google, Facebook, etc)\cite{top_three_issues_centralized_web}, surgem inevitavelmente questões sobre para que são realmente utilizados esses dados e se o utilizador está realmente ciente que estes estão disponíveis para toda e qualquer utilização que seja proveitosa para essas empresas \cite{facebook_data_hell_medium}. Este problema torna-se crítico quando mentes geniais aliadas a poderes incríveis de processamento conseguem literalmente manipular opiniões e culminar em acontecimentos inéditos \cite{cambridge_analytica}.

A descentralização da web é um tema que assenta também em como tratar estes dados, não estivesse o tema aliado ao conceito da nova “World Wide Web”, e de entre outros projetos destacar-se o projeto “Solid”, fundado por Tim Berners-Lee. O projeto Solid está ainda numa fase de desenvolvimento, estando, portanto, aberto a contribuições que possam de facto garantir melhorias e impacto positivo para o futuro \cite{why_web_decentralization_future}.

\section{Motivação}
A descentralização da web não é um tema consensual, e de certo vai contra muitos modelos de negócio actuais. Porém, assim como protegemos as nossas propriedades, devemos proteger os nossos dados, os seus acessos e respetivas utilizações.

O paradigma actual gira em torno de oferecer serviços em troca de publicidade segmentada, com a justificação de sustentar custos de infraestrutura e de desenvolvimento \cite{top_three_issues_centralized_web}. Com um novo paradigma de web descentralizada poderiam ser cortados os custos de infraestrutura, porque deixariam de ser necessários "back-ends", e as aplicações cliente utilizam apenas recursos do cliente. Restariam os custos desenvolvimento, que poderiam eventualmente ser sustentados pela extinção do conceito "gratuito" nas empresas de base digital \cite{why_web_decentralization_future}.

É possível que este tema seja apenas um "não tema" e que não seja possível de implementar massivamente, mas todo o contexto por trás e a quantidade de dados sensíveis (saúde, seguros, etc) que circulam todos os segundos nos trâmites do actual paradigma da web,
merece que sejam dedicados esforços e que pelo menos sirva de alavanca a uma reformulação profunda daquilo que é a web que conhecemos hoje \cite{why_web_decentralization_future}.

\section{Objectivos}
Os objetivos desta dissertação passam por explorar o tema da descentralização da web. O estudo irá ter em conta as alternativas mais relevantes apresentadas neste contexto até ao dia de hoje. Destas alternativas, serão trabalhadas soluções de escalabilidade para o projeto que se apresentar mais promissor.
Passando, desta forma, o objetivo também pela contribuição para um projeto open-source com alto potencial de revolucionar a web, sendo esta contribuição no sentido de trabalhar sobre o mesmo numa prespetiva mais global com vista ao incremento da sua escalabilidade, através da migração para uma arquitetura orientada a micro-serviços, o baixo acoplamento, alta coesão e, consequentemente, a gestão rápida dos diferentes componentes.

\section{Estrutura da Tese}

Esta dissertação é composta pelos seguintes capítulos.

No capítulo \ref{cap:1} (Introdução) elabora-se uma introdução contextualizada ao tema, de forma a que o projeto e objetivos sejam melhor compreendidos.

O capítulo \ref{cap:2} (Estado da Arte) dedica-se ao estado da arte, no sentido de dar a entender as alternativas mais relevantes existentes até ao momento na descentralização da web.

O capítulo \ref{cap:3} (Análise de Valor) incide na proposta de valor assente nos diferentes projetos referentes à descentralização da web apresentados no estado da arte.

O capítulo \ref{cap:4} (Desenho) apresenta uma proposta de desenho para uma possível solução. Tendo em conta que esta tese se desenrolará sobre uns dos projetos de descentralização da web apresentados.

No capítulo \ref{cap:5} (Implementação) apresentam-se alguns pormenores e linhas gerais da implementação proposta.

O capítulo \ref{cap:6} (Avaliação e Experimentação) dedica-se à avaliação da proposta e verificação dos resultados obtidos.

Por fim, o capítulo \ref{cap:7} (Conclusão) apresenta as conclusões tiradas, assim como as adversidades e possibilidades de desenvolvimentos futuros no sentido de dar continuidade ao projeto atual.