% ************ Chapter 1 ************
%\renewcommand{\chaptername}{Chapter}
\chapter{Introdução}
\label{cap:1}
\emph{Este capítulo incide na apresentação contextualizada do tema. Apresentando tópicos como, contextualização, motivação, objetivos e estrutura da dissertação}


\section{Contextualização}
Numa era em que o principal ativo das empresas é a informação, a ponto de existirem empresas que trocam serviços altamente valiosos por informação pessoal (ex: google, facebook, etc), surgem inevitavelmente questões sobre para são realmente utilizados esses dados e se o utilizador está realmente ciente que estes estão disponíveis para toda e qualquer utilização que seja proveitosa para essas empresas. Este problema torna-se crítico quando mentes geniais aliadas a poderes incríveis de processamento conseguem literalmente manipular opiniões e culminar em acontecimentos inéditos.

A descentralização da web é um tema que assenta também em como tratar estes dados, não estivesse o tema aliado ao conceito da nova “World Wide Web”, e de entre outros projetos destacar-se o projeto “Solid”, fundado por Tim Berners-Lee. O projeto Solid está ainda numa fase de desenvolvimento, estando, portanto, aberto a contribuições que possam de facto garantir melhorias e impacto positivo para o futuro.

\section{Motivação}
A descentralização da web não é um tema consensual, e de certo vai contra muitos modelos de negócio actuais. Porém, assim como protegemos as nossas propriedades, devemos proteger os nossos dados, os seus acessos e respetivas utilizações.

O paradigma actual gira em torno de oferecer serviços em troca de publicidade segmentada, com a justificação de sustentar custos de infraestrutura e de desenvolvimento. Com um novo paradigma de web descentralizada poderiam ser cortados os custos de infraestrutura, porque deixariam de ser necessários "back-ends", e as aplicações cliente utilizam apenas recursos do cliente. Restariam os custos desenvolvimento, que poderiam eventualmente ser sustentados pela extinção do conceito "gratuito" nas empresas de base digital.

É possível que este tema seja apenas um "não tema" e que não seja possível de implementar massivamente, mas todo o contexto por trás, merece que sejam dedicados esforços e que pelo menos sirva de alavanca a uma reformulação profunda daquilo que é a web que conhecemos hoje.

\section{Objectivos}
Os objetivos desta dissertação passam por explorar o tema da descentralização da web. O estudo irá ter em conta as alternativas mais relevantes apresentadas neste contexto até ao dia de hoje. Destas alternativas, serão trabalhadas soluções de escalabilidade para o projeto que se apresentar mais promissor.

\section{Estrutura da Tese}

Esta dissertação é composta pelos seguintes capítulos.

No Capítulo 1 elabora-se uma introdução contextualizada ao tema, de forma a que o projeto e objetivos sejam melhor compreendidos.

O Capítulo 2 dedica-se ao estado da arte, no sentido de dar a entender as alternativas mais relevantes existentes até ao momento na descentralização da web.

O Capítulo 3 incide na proposta de valor assente nos diferentes projetos referentes à descentralização da web apresentados no estado da arte.

O Capítulo 4 apresenta uma proposta de desenho para uma possível solução. Tendo em conta que esta tese se desenrolará sobre uns dos projetos de descentralização da web apresentados.
%, este capitulo terá essencialmente duas subdivisões: explicação da arquitetura actual e explicação de duas arquituras propostas.

No Capítulo 5 apresentam-se alguns pormenores e linhas gerais da implementação proposta.

O Capítulo 6 dedica-se à avaliação da proposta e verificação dos resultados obtidos.

Por fim, o Capítulo 7 apresenta as conclusões tiradas, assim como as adversidades e possibilidades de desenvolvimentos futuros no sentido de dar continuidade ao projeto atual.