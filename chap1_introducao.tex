% ************ Chapter 1 ************
\chapter{Introdução}
\label{cap:1}

Numa era em que o principal ativo das empresas é a informação, a ponto de existirem empresas que trocam serviços altamente valiosos por informação pessoal (ex: Google, Facebook, etc)\cite{top_three_issues_centralized_web}, surgem inevitavelmente questões sobre para que são realmente utilizados esses dados e se o utilizador está realmente ciente que estes estão disponíveis para toda e qualquer utilização que seja proveitosa para essas empresas\cite{facebook_data_hell_medium}.

Este problema torna-se crítico quando mentes geniais aliadas a poderes incríveis de processamento conseguem, literalmente, manipular opiniões que culminam em acontecimentos inéditos\cite{cambridge_analytica}.

A descentralização da \emph{Web} é um tema que assenta também em como tratar estes dados, não estivesse o tema adjacente ao conceito da nova “World Wide Web”, e de entre outros projetos destaca-se o “Solid”, fundado por Tim Berners-Lee. O \emph{Solid} está ainda numa fase de desenvolvimento, estando, portanto, aberto a contribuições que possam de facto garantir melhorias e impacto positivo para o futuro\cite{why_web_decentralization_future}.

Este projeto mantém atualmente duas ramificações, a original e uma outra desenvolvida e mantida pela \emph{spinoff} denominada \emph{inrupt} fundada também por Tim Berners-Lee. Esta última ramificação tem base a original mas com uma vertente mais comercial e uma visão mais ambiciosa no sentido de proliferar a utilização do Solid.

\section{Motivação}
A descentralização da \emph{Web} não é um tema consensual, e certamente vai contra muitos modelos de negócio atuais. Porém, assim como se protege as nossas propriedades, devem ser protegidos os nossos dados, os seus acessos e respetivas utilizações.

O paradigma atual gira em torno de oferecer serviços em troca de publicidade segmentada, com a justificação de sustentar custos de infraestrutura e de desenvolvimento\cite{top_three_issues_centralized_web}. Com um novo paradigma de \emph{Web} descentralizada poderiam ser cortados os custos de infraestrutura, porque deixariam de ser necessários \emph{“back-ends”}, e as aplicações cliente utilizariam apenas recursos do cliente. Restariam os custos desenvolvimento, que poderiam eventualmente ser sustentados pela extinção do conceito “gratuito” nas empresas de base digital\cite{why_web_decentralization_future}.

É possível que este tema seja apenas um “não tema” e que não seja plausível a sua implementação de forma massiva, mas todo o contexto por trás e a quantidade de dados sensíveis (saúde, seguros, etc) que circulam todos os segundos nos trâmites do atual paradigma da \emph{Web},
merece que sejam dedicados esforços e que pelo menos sirva de alavanca a uma reformulação profunda daquilo que é a \emph{Web} que conhecemos hoje\cite{why_web_decentralization_future}.

\section{Objectivos \label{objetivos}}
Os objetivos desta dissertação passam por explorar o tema da descentralização da \emph{Web}. O estudo irá ter em conta as alternativas mais relevantes apresentadas neste contexto até ao dia de hoje. Destas alternativas, serão trabalhadas soluções de escalabilidade para o projeto que se apresentar mais promissor.

Assim, está também subjacente como objetivo a contribuição para um projeto \emph{open-source} com alto potencial de revolucionar a \emph{world wide web}, sendo esta contribuição no sentido de trabalhar numa perspetiva mais global com vista ao incremento da sua escalabilidade, através da migração para uma arquitetura orientada a micro-serviços, o baixo acoplamento, alta coesão e, consequentemente, a gestão rápida dos diferentes componentes.

\section{Hipóteses \label{section_hypothesis}}
De forma a estabelecer um ponto de partida para a solução a desenvolver, apresentam-se as hipóteses que deverão ser corroboradas:
\begin{itemize}
    \item H1 - A solução a desenvolver será mais escalável que a atual.
    \item H2 - A solução a desenvolver, nas mesmas condições de escalamento vertical e horizontal, não deverá ter menos \emph{performance} que a atual.
    \item H3 - A experiência de utilização numa perspetiva de utilizador final não irá sofrer alterações.
\end{itemize}

\section{Questões de Investigação \label{section_investigation_questions}}
Estas hipóteses levantam uma série de questões que deverão servir também por base à investigação adjacente a esta dissertação, das quais destacam-se as seguintes:
\begin{enumerate}
    \item \textbf{Como pode o \emph{Solid} ser escalado horizontal e verticalmente?}
    \begin{itemize}
        \item A escalabilidade horizontal e vertical correspondem a escalar os recursos de determinada instância e incrementar o numero de instâncias, respetivamente. Estas duas abordagens tem custos e benefícios diferentes, que devem pesados nas diferentes circunstâncias em que exista decréscimo de performance na instância atual.
        \item O escalamento horizontal implica que possa existir redundância de determinado serviço ou de todo o monólito no caso de se tratar de uma arquitetura monolítica.
    \end{itemize}
    \item \textbf{É possível existir apenas uma implantação do \emph{Solid} escalada de forma descentralizada e anónima?}
    \begin{itemize}
        \item Esta questão está dependente da anterior questão relativa à possibilidade de escalamento horizontal de forma redundante, isto porque esse seria primeiro passo para tornar isto possível.
        \item Por outro lado, o escalamento descentralizado iria implicar uma mecanismo de deteção de novas instâncias dos diferentes serviços na rede, para que possam servir tráfego.
    \end{itemize}
    \item \textbf{O mecanismo de autenticação utilizado atualmente pode ser mantido tendo em conta a migração para uma arquitetura orientada a micro-serviços?}
    \begin{itemize}
        \item A arquitetura orientada a micro-serviços assenta no princípio de baixo acoplamento entre os diferentes serviços constituintes do sistema, assim o mecanismo de autenticação não deve criar uma forte dependência para o com os restantes serviços\cite{building_microservices:2015}.
    \end{itemize}
\end{enumerate}

\section{Metodologia de Investigação}
Para a investigação, no âmbito do desenvolvimento deste projeto, foi utilizado o método \emph{Design Science Research}. Esta abordagem aplica-se sobretudo a investigações com vista a construir novas soluções ou implantar soluções inovadoras sobre problemas que possam já ter outras resoluções descobertas\cite{design_science_research}.

Este processo inclui geralmente seis passos\cite{design_science_research}:
\begin{enumerate}
    \item Identificar o problema, assim como identificar os desafios da investigação e os potenciais benefícios da solução;
    \item \textbf{Definição} dos objetivos para a solução;
    \item \textbf{Desenho} e desenvolvimento dos artefactos;
    \item \textbf{Demonstração} da resolução do problema com recurso aos artefactos desenvolvidos;
    \item \textbf{Avaliação} da solução, comparando os objetivos com os resultados obtidos;
    \item \textbf{Comunicação} do problema, dos artefactos constituintes da solução, a sua utilidade e a forma como realmente contribuir para criar impacto positivo.
\end{enumerate}

\section{Contribuições da Investigação \label{section_contribuicoes_investigacao}}
O trabalho resultante desta dissertação tem como objetivo criar uma possível solução escalável e capaz de alicerçar a proliferação de uma nova \emph{world Wide Web} descentralizada. Seguem-se algumas contribuições resultantes do mesmo:
\begin{itemize}
    \item \emph{A Study About Web Development Frameworks Focused on Users’ Privacy}\cite{solid_article}
    \item Repositório do \emph{Node Solid Server} segundo arquitetura orientada a micro-serviços proposta\cite{repo_node_solid_server}
    \item Repositório do \emph{OpenID Connect provider} para \emph{Node.js}\cite{repo_oidc_op}.
    \item Repositório do \emph{OpenID Connect relying party client}\cite{repo_oidc_rp}.
    \item Repositório do \emph{OpenID Connect authentication manager}\cite{repo_oidc_auth_manager}.
    \item Repositório do \emph{Solid auth client}\cite{repo_solid_auth_client}.
    \item Repositório da aplicação cliente \emph{Solid} micro-serviços com funcionalidade de gestão de ficheiros\cite{repo_solid_filemanager}
    \item Repositório da aplicação cliente \emph{Solid} monólito com funcionalidades clínicas\cite{repo_steve_clinic}.
\end{itemize}

\section{Estrutura da Tese}

Esta dissertação é composta pelos seguintes capítulos.

No capítulo \ref{cap:1} (Introdução) elabora-se uma introdução contextualizada ao tema, de forma a que o projeto e objetivos sejam melhor compreendidos.

O capítulo \ref{cap:2} (Estado da Arte) dedica-se ao estado da arte, no sentido de dar a entender as alternativas mais relevantes existentes até ao momento na descentralização da \emph{Web}.

O capítulo \ref{cap:3} (Análise de Valor) incide na proposta de valor assente nos diferentes projetos referentes à descentralização da \emph{Web} apresentados no estado da arte.

O capítulo \ref{cap:4} (Desenho) apresenta uma proposta de desenho para uma possível solução, tendo em conta que esta dissertação se debruçará sobre um dos projetos de descentralização da \emph{Web} apresentados.

No capítulo \ref{cap:5} (Implementação) são revelados alguns pormenores e linhas gerais da implementação proposta.

O capítulo \ref{cap:6} (Avaliação e Experimentação) dedica-se à avaliação da proposta e verificação dos resultados obtidos.

Por fim, o capítulo \ref{cap:7} (Conclusão) apresenta as conclusões tiradas, assim como as adversidades e possibilidades de desenvolvimentos futuros no sentido de dar continuidade ao projeto atual.