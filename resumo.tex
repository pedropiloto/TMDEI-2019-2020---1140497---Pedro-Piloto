% ********** Resumo **********
\chapter*{Resumo}
Assim como recentes escândalos (e.g. Cambridge Analytica) mostraram, muitas organizações de base digital recolhem dados de utilizadores, armazenam-nos em locais inacessíveis e utilizam-nos como ativos para gerar lucro. Enquanto isso, os utilizadores perdem completamente a propriedade e o controlo dos seus dados, restando-lhes apenas confiar nas  empresas, nas quais terão, provavelmente, de preencher formulários idênticos e replicar toda a sua informação por múltiplas bases de dados.

De forma a dar seguimento ao desenvolvimento da \emph{Web}, mantendo a privacidade do utilizador, este estudo tem o objetivo de entender e iterar sobre um projeto que atua precisamente no âmbito de reinventar uma \emph{Web} mais transparente e centrada no utilizador. A par deste estudo serão exploradas alternativas, no sentido de perceber aquela que está mais orientada para servir como alternativa à actual \emph{Web}.

O projeto com maior destaque nesta dissertação é o \emph{Solid}, este foi fundado por Tim Berners-Lee e conta com uma comunidade forte que dedica os seus esforços a criar contribuições para aquele que é um dos projetos mais promissores neste ramo. 

No decorrer desta dissertação são exploradas as suas potencialidades mas também as limitações actuais de escalabilidade derivadas da sua arquitetura monolítica. De forma a mitigar estas limitações de escalabilidade, o trabalho foca-se em detalhar a migração para uma solução orientada a micro-serviços, modelando as diferentes alternativas possíveis, bem como justificando as decisões arquiteturais mais relevantes.

Assim, a presente dissertação prende-se não só com o estudo do tema de descentralização da \emph{Web} mas também em criar uma contribuição positiva e clara no sentido de mitigar os problemas de escalabilidade do sistema \emph{Solid}.

\textbf{Palavras-Chave}: decentralization, data, storage, authentication, user privacy, micro-services
% ********** End of Resumo **********