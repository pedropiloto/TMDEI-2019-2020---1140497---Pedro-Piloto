% ********** Resumo **********
\chapter*{Resumo}
Assim como recentes escândalos mostraram, muitas empresas de base digital recolhem dados de utilizadores, armazenam-nos em locais inacessíveis e utilizam-nos como ativos para gerar lucro. Enquanto isso, os utilizadores perdem completamente a propriedade e o controlo dos seus dados, restando-lhes apenas confiar nas  empresas em que terão provavelmente de replicar todos os dados.

De forma a dar seguimento ao desenvolvimento da Web, mantendo a privacidade do utilizador, este estudo visa entender e iterar sobre um projeto que actua precisamente no âmbito de reinventar uma \emph{web} mais transparente e centrada no utilizador. A par deste estudo serão exploradas alternativas, no sentido de perceber aquela que está mais orientada para servir como alternativa à actual web, consequentemente lidar com uma adoção massiva.

O projeto com maior destaque nesta dissertação é o \emph{Solid}, este foi fundado por Tim Berners-Lee e conta com uma comunidade forte que dedica os seus esforços a criar contribuições para aquele que é um dos projetos mais promissores neste ramo. No decorrer desta dissertação são exploradas as suas potencialidades mas também as limitações actuais de escalabilidade derivadas da sua arquitetura monolítica, tornando-se este um dos principais focos do trabalho.

De forma a mitigar as limitações de escalabilidade, o trabalho foca-se em detalhar a migração da solução atual para uma orientada a micro-serviços, apontando e modelando as diferentes alternativas possíveis, bem como justificando as decisões arquiteturais mais relevantes. Numa fase mais avançada do desenvolvimento, a solução é    satisfatoriamente testada contra as hipóteses previamente estabelecidas.

Assim, a presente dissertação prende-se não só com o estudo do tema de descentralização da \emph{web} mas também em criar uma contribuição positiva e clara no sentido de mitigar os problemas de escalabilidade do sistema \emph{Solid}.
% ********** End of Resumo **********